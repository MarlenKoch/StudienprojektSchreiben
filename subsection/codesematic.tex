\documentclass[../main.tex]{subfiles}
\begin{document}

In \ref{test1} konnten sechs von zehn Fragen richtig beantwortet werden.
Zur weiteren Untersuchungen wurde \ref{test2} durchgeführt.
In diesem Test war der Prozess des Indexing und der Suche gleich, doch wurde das Test Projekt insofern angepasst, dass sämtlich Dokumentation gelöscht wurde und Namen von Variablen nicht deskreptiv, z.B. \enquote{logger} zu \enquote{l}, waren.
Die vorgestellte Indexing und Such Metohde konnte dieser Änderung nicht standhalten und nur noch vier von zehn Fragen richtig beantworten.
Außerdem ist anzumerken, dass die Fragen, die richtig beantwortet wurden, jene sind, bei dem der dazugehöriges Codeabschnitt weiterhin viele descreptive Namen enthält, da dort externe Biblotheken genutzt werden, dessen Namen nicht angepasst werden können.


Aus diesen beiden Punkt lässt sich vermuten, dass bei der Erzeugung der Embeddings vorallem aus der Namensgebung Information gezogen wird und nicht aus der Logik des Code.

\end{document}