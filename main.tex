% !TeX TXS-program:bibliography = txs:///biber
\documentclass[
	12pt, %Schriftgröße
	a4paper,
	bibliography=totoc, %Inhaltsverzeichniseinträge f+r Quellen
	numbers=noenddot, %Entfernt Punkt hinter Gliederungsnummern
	ngerman, %Sprachpaket
	headsepline, %Headertrennlinie
	%footsepline, %Footertrennlinie
	oneside %einseitiges Druckformat %%% Unterdrücken der leeren Seite nach Titelblatt
	]{scrbook} %Dokumentenklasse (Koma-Script)

\usepackage{microtype}
\usepackage[T1]{fontenc}
\usepackage{float}
\usepackage[utf8]{inputenc}
\usepackage[ngerman]{babel}
\usepackage{url}
\usepackage{graphicx} %Bilder einfügen
%\usepackage{hyphenat}
%\usepackage{pdfpages} %PDF einfügen
\usepackage[a4paper, margin=1in]{geometry}
\usepackage[right]{eurosym} %Euro-Zeichen
\usepackage{amssymb}
\usepackage{setspace} % Zeilenabstand
\usepackage[ 
   colorlinks,        % Links ohne Umrandungen in zu wählender Farbe 
   linkcolor=black,   % Farbe interner Verweise 
   filecolor=black,   % Farbe externer Verweise 
   citecolor=black,   % Farbe von Zitaten 
   urlcolor=blue	  % Farbe von Links
   ]{hyperref} %Verlinkungen
\usepackage[figure]{hypcap}
\usepackage[ngerman]{translator}
\usepackage{blindtext} % Lorem-Ipsum-Plugin
\usepackage[acronym, nonumberlist, toc, translate=babel]{glossaries} %% Glossar mit Akronymen laden und mit babel auf deutsch anzeigen
%\usepackage[
	%nonumberlist, %keine Seitenzahlen anzeigen
	%acronym,      %ein Abkürzungsverzeichnis erstellen
	%toc,          %Einträge im Inhaltsverzeichnis
	%section      %im Inhaltsverzeichnis auf section-Ebene erscheinen
	%]
%{glossaries}

\usepackage{csquotes}
\usepackage[]{bookmark}
\usepackage{scrhack}
\usepackage{tabularx}
\usepackage{listings,xcolor} %Codeanzeige
\usepackage[normalem]{ulem}
\useunder{\uline}{\ul}{}

\usepackage{chngcntr}
\counterwithout{figure}{chapter}
\counterwithout{table}{chapter}

\usepackage{subfiles}
\usepackage{pgfmath}

\newcounter{mylabelcounter}

\makeatletter
\newcommand{\labeltext}[2]{%
  \@bsphack
  \csname phantomsection\endcsname % in case hyperref is used
  \def\@currentlabel{#1}{\label{#2}}%
  \@esphack
}
\makeatother
 
\usepackage[backend=biber]{biblatex}
\addbibresource{lit.bib}

\definecolor{dkgreen}{rgb}{0,.6,0}
\definecolor{dkblue}{rgb}{0,0,.6}
\definecolor{dkyellow}{cmyk}{0,0,.8,.3}

\lstset{
    % numbers=left, 
    % numberstyle=\tiny, 
    % numbersep=5pt,
    breaklines=true,
    frame=lines,
    % language=sh,
    basicstyle=\small\ttfamily,
    columns=flexible,
    showstringspaces=false
    % basicstyle=\ttfamily\fontsize{10}{12}\selectfont,
    % keywordstyle    = \color{dkblue},
    % stringstyle     = \color{red},
    % identifierstyle = \color{dkgreen},
    % commentstyle    = \color{gray},
    % emph            =[1]{php},
    % emphstyle       =[1]\color{black},
    % emph            =[2]{if,and,or,else},
    % emphstyle       =[2]\color{dkyellow}
} 

%%%%%%%%%%%%%%%%%%%%%%%%%%%%%%%%%%%%%%%%%%%%%%%%%%%%%
%%%%%%%%%%% Sonderformatierung
%%%%%%%%%%%%%%%%%%%%%%%%%%%%%%%%%%%%%%%%%%%%%%%%%%%%%

% Seitenabstände definieren
\geometry{verbose,tmargin=3cm,bmargin=2cm,lmargin=3cm,rmargin=3cm} 

% Hurenkinder und Schusterjungen verhindern (Ja, das heißt wirklich so!!!)
\clubpenalty= 10000
\widowpenalty= 10000
\displaywidowpenalty= 10000 

%% Bei Referenzen im Text wird jetzt bei allen Ebenen "Kapitel" vorgestellt, z.b. Kapitel 2, Kapitel 2.2, Kapitel 6.3.2
\addto\extrasngerman{
    \def\sectionautorefname{Kapitel}%
    \def\subsectionautorefname{Kapitel}%
    \def\subsubsectionautorefname{Kapitel}%
    }

% Vertikaler Abstand zwischen Ende Textblock - Ende Fußzeile --> Abstand der Seitenzahl von Rand erhöhen 
\setlength{\footskip}{10mm}

% Abstand vor/nach Überschriften verändern

\RedeclareSectionCommand[%
    beforeskip=0.5\baselineskip,
    afterskip=0.5\baselineskip,
]{chapter}

\RedeclareSectionCommand[%
    beforeskip=0.5\baselineskip,
    afterskip=0.5\baselineskip,
]{section}

\RedeclareSectionCommand[%
    beforeskip=0.1\baselineskip,
    afterskip=0.1\baselineskip,
]{subsection}

\RedeclareSectionCommand[%
    beforeskip=0.01\baselineskip,
    %%afterskip=0.2\baselineskip
]{paragraph}

\setlength{\abovecaptionskip}{4pt}  % 1pc=12pt 
\setlength{\belowcaptionskip}{0pt}
%\setlength{\textfloatsep}{4pt}
\setlength{\intextsep}{1pc}

%% Verkleinerung der Textgröße unter Abbildungen
\addtokomafont{caption}{\small}



% Den Punkt am Ende der Glossareinträge deaktivieren
\renewcommand*{\glspostdescription}{}

%Glossar-Befehle anschalten
\makenoidxglossaries{}
\loadglsentries{glossar.tex}

% sorgt dafür, dass bei Leerzeile die Einrückung verhindert und stattdessen eine Leerzeile eingefügt wird % erspart bigskips und erhöht die Lesbarkeit im LaTeX-Text 
\KOMAoptions{parskip=full*}

% ändert Titelschriftart in Serifen-Normalschriftart
\addtokomafont{disposition}{\rmfamily} 




%%%%%%%%%%%%%%%%%%%%%%%%%%%%%%%%%%%%%%%%%%%%%%%%%%%%%
\newcommand{\studentNameOne}{Marlen Koch}
\newcommand{\studentNameTwo}{Amelie Hoffmann}
\newcommand{\matnrOne}{123}
\newcommand{\matnrTwo}{456}
\newcommand{\type}{Studienprojekt}
\newcommand{\topic}{Implementierung und Evaluation eines KI-basierten Schreibassistenten im akademischen Umfeld}
\newcommand{\studiengangh}{Informatik}
\newcommand{\fachbereich}{FB2: Duales Studium --- Technik}
\newcommand{\studiengang}{Informatik}
\newcommand{\company}{SAP SE}
\newcommand{\betreuerHS}{Gert Faustmann}
\newcommand{\jahrgang}{2023}
%%%%%%%%%%%%%%%%%%%%%%%%%%%%%%%%%%%%%%%%%%%%%%%%%%%%%>>>>>>>


% \newcommand{\quickwordcount}[1]{%
%   \immediate\write18{texcount -1 -sum -merge -q -relaxed #1.tex > #1-words.sum }%
%   \input{#1-words.sum}%
% }

\pgfkeys{/pgf/number format/.cd, use comma=false, 1000 sep={}}

\newcommand{\quickwordcount}[1]{%
  \immediate\write18{texcount -1 -sum -merge -q -relaxed #1.tex > #1-words.sum }%
  \newread\wordfile
  \openin\wordfile=#1-words.sum
  \read\wordfile to \wordcount
  \closein\wordfile
  \pgfmathparse{\wordcount-1000}%
  \pgfmathprintnumber{\pgfmathresult}%
}
%%%%%%%%%%%%%%%%%%%%%%%%%%%%%%%%%%%%%%%%%%%%%%%%%%%%%%%%%%%%%%

\begin{document}

% falsche Default-Silbentrennung überschreiben
%\include{hyphenation}

% keine Silbentrennung sonder space zwischen den Wörtern nutzen, wie Blick Absatz in word 
\spaceskip=0.3em plus 4em minus 0.2em

%%%%%%%%%%%%%%%%%%%%%%%%%%%%%%%%%%%%%%%%%%%%%%%%%%%%%>>>>>>>
%%%%%%%%%%% Titelblatt

%% Anordnung und Aussehen von Titel und Untertitel

\subject{\type}

\title{
\normalfont\endgraf\rule{\textwidth}{.4pt}
\begingroup
	\centering
	\linespread{1.5}
	\huge\topic%
\endgroup
\endgraf\rule{\textwidth}{.4pt}
}
 
%%Eigentlich nicht besonders schön, aber Koma erlaubt die Anordnung eines weiteren Felder (hier: Fachbereich) nicht
\date{\vspace{-2cm}\normalsize vorgelegt am \today \vspace{1cm}}
%% \date muss leer angegeben werden, um die Default-Datumsanzeige

\publishers{
	\begin{tabular}{l l}
	\textbf{\normalsize{Namen:}} & \normalsize{\studentNameOne, \studentNameTwo}  \tabularnewline%
    % \textbf{\normalsize{Matrikelnummern:}} & \normalsize{\matnrOne,\matnrTwo}  \tabularnewline%
	\textbf{\normalsize{Ausbildungsbetrieb:}} & \normalsize{\company}  \tabularnewline%
    \textbf{\normalsize{Fachbereich:}} & \normalsize{\fachbereich} \tabularnewline%
    \textbf{\normalsize{Studienjahrgang:}} & \normalsize{\jahrgang} \tabularnewline%
	\textbf{\normalsize{Studiengang:}} & \normalsize{\studiengang}  \tabularnewline%
    \textbf{\normalsize{Betreuer Hochschule:}} & \normalsize{\betreuerHS} \tabularnewline%
	\textbf{\normalsize{Wortanzahl:}} & \normalsize{\quickwordcount{main}}
	\end{tabular}
    

}


\titlehead{\begin{center}
    \includegraphics[scale=0.7]{bilder/header_logo.PNG}
    \end{center}
    }

\maketitle

% \onehalfspacing


\vspace{20mm}

\begin{tabular}{lp{2em}l} 
    \hspace{4cm}   && \hspace{4cm} \\\cline{1-1}\cline{3-3} 
    Ort, Datum     && \studentNameOne{}
\end{tabular}

\begin{tabular}{lp{2em}l} 
    \hspace{4cm}   && \hspace{4cm} \\\cline{1-1}\cline{3-3} 
    Ort, Datum     && \studentNameTwo{}
\end{tabular}

% \vspace{\fill}
% \normalsize{Von der betrieblichen Betreuung zur Kenntnis genommen:}
% \vspace*{20mm}

% \begin{tabular}{lp{2em}l} 
%     \hspace{4cm}   && \hspace{4cm} \\\cline{1-1}\cline{3-3} 
%     Ort, Datum     && \betreuerUnt
% \end{tabular}
% \vspace*{20mm}

% \begin{tabular}{lp{2em}l} 
%     \hspace{4cm}   && \hspace{4cm} \\\cline{1-1}\cline{3-3} 
%     Ort, Datum     && Peter, Jenny
% \end{tabular}

%%%%%%%%%%%%%%%%%%%%%%%%%%%%%%%%%%%%%%%%%%%%%%%%%%%%%%%%%%%%%%%%%%%%%%%%%%%%%%%%%%%%%%%%%%%%%%%%%%%%%%%%%%%%%%%%%%%%%%%%%%%
%%%%%%%%%%% Dokumenteninhalt START
%%%%%%%%%%%%%%%%%%%%%%%%%%%%%%%%%%%%%%%%%%%%%%%%%%%%%%%%%%%%%%%%%%%%%%%%%%%%%%%%%%%%%%%%%%%%%%%%%%%%%%%%%%%%%%%%%%%%%%%%%%%

%%%%%%%%%%%%%%%%%%%%%%%%%%%%%%%%%%%%%%%%%%%%%%%%%%%%%
%%%%%%%%%%% Abstract
\chapter*{Kurzfassung}
\addcontentsline{toc}{chapter}{Kurzfassung}
\subfile{chapter/abstract.tex}

\pagenumbering{Roman} % römische Seitenzahlen

%%%%%%%%%%%%%%%%%%%%%%%%%%%%%%%%%%%%%%%%%%%%%%%%%%%%%
%%%%%%%%%%% Inhaltsverzeichnis, Tabellen, Abbildungen, etc.
\newpage

\tableofcontents{}
\addcontentsline{toc}{chapter}{Inhaltsverzeichnis}

\listoffigures
\addcontentsline{toc}{chapter}{Abbildungsverzeichnis}
\newpage

% \listoftables
% \addcontentsline{toc}{chapter}{Tabellenverzeichnis}

\section*{Hinweise}

Aus Gründen der besseren Lesbarkeit wird im Text verallgemeinernd das generische Maskulinum verwendet. Diese Formulierungen umfassen gleichermaßen weibliche, männliche und diverse Personen.\\

Um den Lesefluss zu verbessern, werden Quellen, die sich auf einen gesamten Absatz beziehen, am Ende des Absatzes nach dem Schlusspunkt angegeben, während Abbildungen, Codebeispiele und Tabellen, die den
Lesefluss stören, im Anhang platziert werden, auf den im Text zusätzlich verwiesen wird.
\clearpage

%Tabellenverzeichnis, falls nötig: 

%\listoftables
%\addcontentsline{toc}{chapter}{Tabellenverzeichnis}


\printnoidxglossaries{}

\clearpage

%% arabische Seitenzahlen
\pagenumbering{arabic}

%%%%%%%%%%%%%%%%%%%%%%%%%%%%%%%%%%%%%%%%%%%%%%%%%%%%%
%%%%%%%%%%% Kapitel

\chapter{Einleitung}\label{ch:einleitung}
\subfile{chapter/einleitung.tex}

\chapter{Technische Hintergründe}\label{ch:Technische Hintergründe}
\subfile{chapter/theoretischenGrundlagen.tex}

\chapter{Wissenschaftliches Schreiben und Künstliche Intelligenz}\label{ch:wissenschaftliches Schreiben und KI}
\subfile{chapter/wissenschaftlichesSchreibenUndKI.tex}

\chapter{Anforderungen}\label{ch:Anforderungen}
\subfile{chapter/anforderungen.tex}

\chapter{Implementierung}\label{ch:Implementierung}
\subfile{chapter/implementierung.tex}

\chapter{Bewertung}
\subfile{chapter/bewertung.tex}

\chapter{Fazit}\label{ch:fazit}
\subfile{chapter/fazit.tex}
%%%%%%%%%%%%%%%%%%%%%%%%%%
% Quellen
%%%%%%%%%%%%%%%%%%%%%%%%%

\spaceskip=0.3em plus 0.5em minus 0.2em
\printbibliography

%% \bibliographystyle{alpha} %% tu es nicht, niemals, das ist eklig, nicht einkommentieren

\chapter*{Ehrenwörtliche Erklärung}
\addcontentsline{toc}{chapter}{Ehrenwörtliche Erklärung}
\subfile{chapter/ehrenwörtliche_erklärung.tex}

\end{document}

