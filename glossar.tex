%%%%%%%%%%%%%%%%%%%%%%%%%%%%%%%%%%%%%%%%%%%%%%%%%%%%%%%%%%%%%%%
%%%%%%%%%%%%%%%%%%%%%%ABKÜRZUNGEN
\newacronym{ki}{KI}{Künstliche Intelligenz}
\newacronym{llm}{LLM}{Large Language Model}
\newacronym[
  longplural={Neuronale Netze},   
  shortplural={NNs}               
]{nn}{NN}{Neuronales Netz}
\newacronym{mlp}{MLP}{Multi-Layer-Perceptron}
\newacronym{api1}{API}{ \glslink{glos:api}{Application Programming Interface}}
\newacronym{css}{CSS}{Cascading Style Sheets}
\newacronym{sql}{SQL}{Structured Query Language}

%%%%%%%%%%%%%%%%%%%%%%%%%%%%%%%%%%%%%%%%%%%%%%%%%%%%%%%%%%%%%%%
%%%%%%%%%%%%%%%%%%%%%%Glossar

\newglossaryentry{glos:llm}{
  name={Large Language Model},
  description={Ein künstliches neuronales Netzwerkmodell, das für die Sprachverarbeitung trainiert wurde und Aufgaben wie Textgenerierung, Übersetzung oder Zusammenfassung beherrscht. Beispiele sind GPT-3 oder GPT-4}
}

% \newglossaryentry{glos:prompt}{
%   name={Prompt},
%   description={Texti für Chatty}
% }

\newglossaryentry{glos:ki}{
    name={Künstliche Intelligenz},
    description={Ein informationsverarbeitendes Computersystem, das mithilfe nicht-strikt-deterministischer Modelle oder Algorithmen automatisiert Eingabedaten analysiert und daraus Ausgaben erzeugt, um komplexe Aufgaben wie Prognosen, Klassifikationen oder Generierungen zu unterstützen. Wird im Folgenden synonym mit LLM verwendet}
}


\newglossaryentry{glos:nn}{
    name={Neuronales Netz},
    description={Ein aus mehreren Schichten von Neuronen bestehendes Netzwerk, welches dem Gehirn nachempfunden ist}
}

% \newglossaryentry{large-language-model}{
%     name={Large Language Model},
%     description={Ein KI-Modell, welches mit einer Vielzahl von Daten trainiert wurde und Text generieren kann.}
% }

\newglossaryentry{glos:prompt}{
    name={Prompt},
    description={Text, welcher einem LLM als Eingabe dient}
}

\newglossaryentry{open-source}{
    name={Open Source},
    description={Software, deren Quellcode öffentlich zugänglich ist}
}

\newglossaryentry{glos:api}{
    name={Application Programming Interface},
    description={Schnittstelle zwischen zwei Anwendungen, die es ihnen erlaubt, miteinander zu kommunizieren}
}

\newglossaryentry{parameter}{
    name={Parameter},
    description={Hier: Verstellbare Gewichtungen eines NNs oder LLMs, welche durch das Training festgelegt werden. Je mehr Parameter ein Modell hat, desto qualitativer sind im Allgemeines seine Antworten}
}

\newglossaryentry{chatgpt}{
    name={ChatGPT},
    description={Ein LLM von der Firma OpenAI}
}

\newglossaryentry{backpropagation}{
    name={Backpropagation},
    description={Eine Trainingsmethode für NNs, bei der die Gewichte gradiell anhand der Ergebnisse angepasst werden}
}

\newglossaryentry{transformer}{
    name={Transformer},
    description={Die Transformer-Architektur bezeichnet die Architektur einer KI, welche eine Eingabe in einem bestimmten Format in eine Ausgabe eines anderen Formates umwandelt}
}

\newglossaryentry{token}{
    name={Token},
    description={Die kleinstmöglichen Textbausteine, welche ein LLM verarbeiten kann}
}

\newglossaryentry{halluzinationen}{
    name={Halluzinationen},
    description={Hier: Aussagen eines LLMs, welche plausibel klingen, aber inkorrekt sind}
}

\newglossaryentry{ollama}{
    name={Ollama},
    description={Eine lokal laufende Plattform für LLMs, welche man vom Terminal aus steuern kann}
}

\newglossaryentry{tooltips}{
    name={Tooltips},
    description={Ein kleines Pop-Up, welches erscheint, wenn der Mauszeiger über ein Element bewegt wird oder es angeklickt wird}
}

\newglossaryentry{augmented-intelligence}{
    name={Augmented Intelligence},
    description={Ein KI-basiertes Hilfsmittel, welches nur unterstützend wirkt, jedoch keine Verantwortung (in dem Fall das alleinige Verfassen eines Textes) übernimmt}
}

\newglossaryentry{prompt-engineering}{
    name={Prompt Engineering},
    description={Das gezielte Verändern von KI-Eingaben, um näher an das gewünschte Ergebnis heranzukommen}
}

\newglossaryentry{softmax}{
    name={Softmax},
    description={Funktion, welche mehrere reelle Zahlen als Eingabe erhält und die gleiche Menge an Zahlen als Ausgabe zurückgibt. Die einzelnen Werte liegen zwischen 0 und 1 und die Summe ist 1.}
}