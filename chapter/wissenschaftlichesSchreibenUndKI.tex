\documentclass[../main.tex]{subfiles}
\begin{document}

Mehrere Studien betonen die Bedeutung des wissenschaftlichen Schreibens für die persönliche und akademische Entwicklung 
von Schülern und vor allem Studenten\cite{influencingUsingAi,ZukunftWissenschaftlichesPublizieren}. Dabei geht es ebenso um das Erlernen der Fähigkeit, sich verständlich auszudrücken und 
eigene Ideen und Gedanken strukturiert vermitteln zu können, wie auch um die Kompetenz, sich durch Recherche neues Wissen 
anzueignen, dieses zu strukturieren und in andere Zusammenhänge zu bringen. Zudem sind auch Organisation und Planung 
wichtige Fähigkeiten, die zum erfolgreichen Verfassen eines wissenschaftlichen Textes benötigt werden.\cite{SelfEfficacyBeliefs} 

Für den entstehenden Text sind sowohl der Inhalt als auch die Struktur relevant. Der Inhalt bildet den Kern des Textes. Er beinhaltet die Ideen und Gedanken des Autors und sollte 
informativ, bedeutsam und originell sein. Der Inhalt sorgt dafür, dass der geschriebene Text seinen Zweck erfüllt, beispielsweise 
den Leser über ein Thema zu informieren. \\
Die Struktur des Textes hingegen umfasst die Art des Schreibens, die Anordnung des Inhalts und das Herstellen eines Leseflusses. 
Ein gut strukturierter Text gibt Ideen in zusammenhängender und logischer Reihenfolge wieder. Eine gute Struktur hilft, den Inhalt des
Textes zu vermitteln. Dementsprechend sollte bei dem Prozess des wissenschaftlichen Schreibens der Inhalt wie auch die Struktur des entstehenden 
Textes beachtet werden.\cite{teachers}

Sowohl während der Schulzeit als auch auf dem weiteren Bildungsweg trainieren Schüler und Studenten das Verfassen kohärenter 
wissenschaftlicher Texte und die damit verbundenen Fähigkeiten. Dabei stellt nicht nur die nötige Recherche sondern auch das Schreiben selbst
häufig eine Herausforderung dar. Wie KI eingesetzt werden kann, um Schüler und Studenten bei diesen Aufgaben zu 
unterstützen, soll in den folgenden Abschnitten erläutert werden. 

\section{Self-Efficacy}

Eine Anfang 2025 veröffentlichte Studie von Yulu Cui untersucht die Gründe, weshalb sich Studenten für die Nutzung von 
KI bei dem Verfassen akademischer Arbeiten entscheiden. Der Fokus wird vorrangig auf emotionale 
Aspekte gelegt. Besonders personalisiertes Feedback und eine intuitiv zu bedienende Nutzeroberfläche seien entscheidend.\cite{influencingUsingAi} \\
Dabei verweist Cui mehrfach auf das Konzept Self-Efficacy (Selbstwirksamkeit). Selbstwirksamkeit beschreibt in diesem Zusammenhang den Glauben 
an die eigene Fähigkeit, eine Aufgabe erfolgreich zu erledigen. Schätzt ein Student seinen Arbeitsstil als effizient ein, 
verfügt er auch über ein stärkeres Gefühl der Selbstwirksamkeit.\cite{influencingUsingAi,SelfEfficacyBeliefs}\\
Nach van Blankenstein et al. stellt die Aufgabe des wissenschaftlichen Schreibens insbesondere für noch unerfahrene Studenten 
ein große Herausforderung dar. Bereits 1999 stellten Pajares, Miller und Johnson sowie Pajares und Valiante fest,
dass die Befürchtung, beim Schreiben zu versagen oder schlechte Erfahrungen zu machen, sich negativ auf die Schreibleistung 
auswirken.\cite{writingSelfBeliefs,writingSelfBeliefsMiddleSchool}

Das Gefühl, während des Schreibprozesses über eine hohe Selbstwirksamkeit zu verfügen, vereinfache den 
Schreibprozess und führe somit auch zu besseren Ergebnissen.\cite{SelfEfficacyBeliefs} \\
Durch den richtigen Einsatz von KI während verschiedener Schritte des wissenschaftlichen Schreibprozesses erhält der Student
ein Gefühl besserer Effizienz während des Arbeitens, was sich positiv auf das Gefühl der Selbstwirksamkeit und damit auf den Erfolg
beim Schreiben auswirkt.\\
KI kann beispielsweise schnell Feedback zu geschriebenen Textteilen geben. Dies helfe den Studenten, ihre eigene 
Leistung besser einzuschätzen und besser mit negativen Emotionen umzugehen, beziehungsweise diese ganz zu vermeiden. Somit 
können sie sich leichter wieder auf die eigentliche Aufgabe konzentrieren.\cite{SelfEfficacyBeliefs} 



\section{Vor- und Nachteile der Nutzung von Künstlicher Intelligenz}

\subsection{Einsatzmöglichkeiten und Vorteile}

Nicht nur für schnelles Feedback zu bereits geschriebenen Texten kann KI während des Schreibprozesses eingesetzt werden. 
Weitere Verwendungsmöglichkeiten sind das Durcharbeiten von Literatur, das Erstellen von Zusammenfassungen, die Unterstützung bei der Suche nach 
möglichen Forschungsthemen und die Verbesserung des Stils und der Grammatik von bereits geschriebenen Texten. Die genannten Einsatzmöglichkeiten 
tragen dazu bei, die Leistungsfähigkeit während des Schreibprozesses zu erhöhen und somit das Gefühl der Selbstwirksamkeit des Autors zu steigern.\cite{SelfEfficacyBeliefs}

Die Erstellung einer Kurzfassung, welche für jede wissenschaftliche Arbeit notwendig ist, jedoch keine geistige Schöpfungshöhe vom Autor mehr verlangt, 
kann von KI übernommen werden. Ebenso eignet sie sich zur Korrektur von Texten. KI Werkzeuge können komplexe Sprache vereinfachen, falsch 
verwendete Wörter ersetzen und durch das Vorschlagen passender Fachbegriffe und Synonyme die sprachliche Vielfalt eines Textes erhöhen. Durch eine 
automatische Korrektur von Grammatik und Rechtschreibung, sowie Vorschläge zum Umstellen der Satzstrukur, lässt sich der Lesefluss eines Textes verbessern.\cite{ZukunftWissenschaftlichesPublizieren,teachers}\\
Eine weitere Verwendungsmöglichkeit für KI ist das Generieren von Forschungsfragen. Da KI Modelle mit einer großen Wissensbasis trainiert werden, 
können so neue und interdisziplinäre Ideen entstehen.\cite{ZukunftWissenschaftlichesPublizieren,humanWritingToAi}

Eine im Jahr 2023 veröffentlichte Studie zu der Nutzung von KI-Schreibwerkzeugen von indonesischen Lehrkräften stellt fest, dass sich die Qualität der von 
Schülern geschriebenen Texte durch die Nutzung von KI erhöht. Die Texte seien klarer formuliert, enthielten weniger Fehler und 
wirkten allgemein kohärenter. Zudem unterstützten die KI Werkzeuge die Schüler, Schreibblockaden zu überkommen und Ideen für das Schreiben von Texten zu entwickeln.\cite{teachers} 

\subsection{Risiken und Nachteile}

Trotz der genannten Vorteile und Einsatzmöglichkeiten gibt es immer wieder Kritik an der Verwendung künstlicher Intelligenz bei dem Verfassen wissenschaftlicher Texte. 
Die Nutzung von KI-Werkzeugen während des wissenschaftlichen Schreibens verlangt von Studenten vor allem die Kompetenz, generierte Inhalte kritisch zu 
hinterfragen. Mehrere Studien äußern Bedenken, dass die Fähigkeit des kritischen Denkens, sowie Kreativität und Originalität des Autors verloren gehen könnten.\cite{ZukunftWissenschaftlichesPublizieren,teachers,BucherSchwarzerHolzwweißig}

Überdies ergibt sich durch die in Kapitel 2 beschriebene Funktionsweise von KI eine Plagiatsgefahr. KI-Werkzeuge wie ChatGPT generieren einen Text Wort für Wort, wobei das nächste anhand 
einer Liste von Wahrscheinlichkeiten ausgewählt wird. Dabei passiert es häufig, dass ein Text aus den Trainingsdaten exakt oder leicht verändert wiedergegeben wird, wie die New York Times in 
einer Klage gegen OpenAI nachweist\cite{NYTimes}. Es besteht also die Möglichkeit, durch die Verwendung von KI unbeabsichtigt Plagiate zu erstellen. Der deutsche Hochschulverband definiert Plagiate als 
die "`wörtliche und gedankliche Übernahme fremden geistigen Eigentums ohne entsprechende Kenntlichmachung"'.\cite{Hochschulverband} \\ Bei der Nutzung von KI-Werkzeugen fehlt es häufig an Transparenz, 
zu welchen Anteilen der entstandene Text aus neu generierten Inhalten besteht oder lediglich die Kopie von Trainingstexten ist. Daraus ergibt sich die Notwendigkeit, KI generierte Texte nicht nur 
auf den Inhalt, sonder darüber hinaus auf potentielle Plagiate zu prüfen. Dazu gibt es unter anderem KI-Plagiatsdetektoren, welche jedoch nicht immer zuverlässig sind. Besonders wenn ein Text 
leicht verändert wiedergegeben wird, ist dies schwer zu erkennen.\\ Plagiatsvorwürfe können "`zum Nichtbestehen von Prüfungsleistungen, Aberkennung von Abschlüssen  oder  zur  Zwangsexmatrikulation  
führen."'\cite{Plagiate} Aufgrund dieser potentiellen Folgen liegt es im Interesse von Studenten, Plagiatsvorwürfe zu vermeiden. Viele Hochschulen sehen bereits die generelle Verwendung von KI bei Hausarbeiten 
oder ähnlichen Leistungen als Plagiat an, da ein mit KI generierter Text keine eigene Schöpfungshöhe aufweist, aber trotzdem als eigene Leistung ausgegeben wird. Andere erlauben die Verwendung von 
KI mit entsprechender Kennzeichnung.\cite{Plagiate}

Zudem kann der übermäßige Einsatz von KI dazu führen, dass Schüler und Studenten das Verständnis für die eigenen Texte fehlt. Schüler könnten die im 
Kontext verwendeten, von der KI vorgeschlagenen Fachbegriffe beispielsweise nicht mehr verstehen. Sollten sie Synonyme, Fachbegriffe und Formulierungen 
übernehmen, ohne diese zu hinterfragen, kann dies zu einem übertrieben förmlichen und schwer verständlichen Schreibstil führen\cite{teachers}. 
So kann sowohl die Kreativität als auch der persönliche Schreibstil verloren gehen. Da KI Modelle auf Grundlage verschiedener Texte trainiert werden, ist es
schwer, während der Verwendung eines KI-Schreibwerkzeugs zum Schreiben oder Umformulieren von Texten, einen einheitlichen und persönlichen Schreibstil aufrecht zu erhalten\cite{creativeWriting}. Hinzu kommt,
dass KI Modelle, aufgrund ihrer in Kapitel 2 erklärten Funktionsweise, Schwerpunkte nicht nach inhaltlichen, sondern mathematischen Kriterien legen\cite{berensBolk}. Somit kann der 
Einsatz von KI dem Inhalt ebenso wie der Struktur eines Textes schaden.\\

Ein Artikel des Open-Access-Publikationsportals "`German Medical Science"' zu dem Thema "`Künstliche Intelligenz und ChatGPT: Über die Zukunft des wissenschaftlichen Publizierens"'
betont insbesondere die Eigenleistung, welche zum Verfassen einer wissenschaftlichen Arbeit erforderlich ist. Die KI solle lediglich als Assistent dienen, und
nicht die Rolle des Verfassers annehmen, dementsprechend keinen eignen Inhalt generieren. \\ Sollte ein KI Modell verwendet werden, welches keinen Internetzugang hat,
fehlem diesen aktuelle Daten und Forschungsergebnisse. Wird ein nicht lokal laufendes Modell verwendet, kann durch die Verarbeitung der Daten durch den KI Anbieter
der Datenschutz nicht mehr gewährleistet werden.\cite{ZukunftWissenschaftlichesPublizieren} \\ Bucher, Holzweißig und Schwarzer betonen vor allem die Gefahr, dass 
generative KI halluzinieren und damit Falschinformationen hervorbringen kann. So können wissenschaftliche Fakten durch KI verzerrt werden. In ihrem Buch "`Künstliche Intelligenz und wissenschaftliches Arbeiten"' nennen sie drei 
Gründe, welche gegen die Verwendung künstlicher Intelligenz im Studium sprechen: rechtliche sowie ethische Bedenken, der Einsatz von KI wird in der Prüfungsordnung untersagt, 
der Einsatz von KI wird vom Partnerunternehmen im Zuge eines dualen Studiums, primär aus datenschutzrechtlichen Gründen, untersagt.\cite{BucherSchwarzerHolzwweißig} Auch einige Verläge lassen 
die Nutzung von KI zu dem Erstellen wissenschaftlicher Publikationen nicht zu, oder fordern zu mindest eine Kennzeichnung\cite{ZukunftWissenschaftlichesPublizieren}. 

Häufig wird eine Art "`augmented intelligence"' gefordert, also ein Hilfsmittel, welches zwar die Effizienz steigert, aber selber nicht das Verfassen der 
Arbeit und die damit verbundene Verantwortung übernimmt. Schüler und Studenten sollen die generierten Inhalte kritisch hinterfragen und Fakten überprüfen. Die Kreativität und 
Eigenverantwortung des Autors soll erhalten bleiben.\cite{BucherSchwarzerHolzwweißig,humanWritingToAi,teachers,ZukunftWissenschaftlichesPublizieren} 

\section{bereits bestehende KI-Lösungen und Hilfsmittel}


\end{document}