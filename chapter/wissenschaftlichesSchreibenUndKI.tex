\documentclass[../main.tex]{subfiles}
\begin{document}

Mehrere Studien betonen die Bedeutung des wissenschaftlichen Schreibens für die persönliche und akademische Entwicklung 
von Schülern und vor allem Studenten \cite{influencingUsingAi}. Dabei geht es sowohl um das Erlernen der Fähigkeit, sich verständlich auszudrücken und 
eigene Ideen und Gedanken strukturiert vermitteln zu können, als auch um die Fähigkeit, sich durch Recherche neues Wissen 
anzueignen, dieses zu strukturieren und in andere Zusammenhänge zu bringen. Zudem sind auch Organisation und Planung 
wichtige Fähigkeiten, die zum erfolgreichen Verfassen eines wissenschaftlichen Textes benötigt werden. 

Für den dabei entstehenden Text sind sowohl der Inhalt als auch die Struktur relevant. Der Inhalt bildet den Kern des Textes. Er beinhaltet die Ideen und Gedanken des Autors und sollte 
informativ, bedeutsam und originell sein. Der Inhalt sorgt dafür, dass der geschriebene Text seinen Zweck erfüllt, beispielsweise 
den Leser über ein Thema zu informieren. \\
Die Struktur des Textes hingegen umfasst die Art des Schreibens, die Anordnung des Inhalts und das Herstellen eines Leseflusses. 
Ein gut strukturierter Text gibt Ideen in zusammenhängender und logischer Reihenfolge wider. Eine gute Struktur hilft, den Inhalt des
Textes zu vermitteln. Dementsprechend sollte bei dem Prozess des wissenschaftlichen Schreibens sowohl der Inhalt als auch die Struktur des entstehenden 
Textes beachtet werden. 

Sowohl während der Schulzeit als auch auf dem weiteren Bildungsweg trainieren Schüler und Studenten das Verfassen kohärenter 
wissenschaftlicher Texte und die damit verbundenen Fähigkeiten. Dabei stellt sowohl die nötige Recherche als auch das Schreiben selbst
oft eine Herausforderung dar. Wie künstliche Intelligenz eingesetzt werden kann, um Schüler und Studenten bei diesen Aufgaben zu 
unterstützen, soll in den folgenden Abschnitten erläutert werden. 

\section{Self-Efficacy}
\section{Vor- und Nachteile der Nutzung von KI}
\section{bereits bestehende Tools}


\end{document}