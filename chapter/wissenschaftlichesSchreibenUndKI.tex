\documentclass[../main.tex]{subfiles}
\begin{document}

Mehrere Studien betonen die Bedeutung des wissenschaftlichen Schreibens für die persönliche und akademische Entwicklung 
von Schülern und vor allem Studenten\cite{influencingUsingAi,ZukunftWissenschaftlichesPublizieren}. Dabei geht es sowohl um das Erlernen der Fähigkeit, sich verständlich auszudrücken und 
eigene Ideen und Gedanken strukturiert vermitteln zu können, als auch um die Fähigkeit, sich durch Recherche neues Wissen 
anzueignen, dieses zu strukturieren und in andere Zusammenhänge zu bringen. Zudem sind auch Organisation und Planung 
wichtige Fähigkeiten, die zum erfolgreichen Verfassen eines wissenschaftlichen Textes benötigt werden. 

Für den dabei entstehenden Text sind sowohl der Inhalt als auch die Struktur relevant. Der Inhalt bildet den Kern des Textes. Er beinhaltet die Ideen und Gedanken des Autors und sollte 
informativ, bedeutsam und originell sein. Der Inhalt sorgt dafür, dass der geschriebene Text seinen Zweck erfüllt, beispielsweise 
den Leser über ein Thema zu informieren. \\
Die Struktur des Textes hingegen umfasst die Art des Schreibens, die Anordnung des Inhalts und das Herstellen eines Leseflusses. 
Ein gut strukturierter Text gibt Ideen in zusammenhängender und logischer Reihenfolge wider. Eine gute Struktur hilft, den Inhalt des
Textes zu vermitteln. Dementsprechend sollte bei dem Prozess des wissenschaftlichen Schreibens sowohl der Inhalt als auch die Struktur des entstehenden 
Textes beachtet werden.\cite{teachers}

Sowohl während der Schulzeit als auch auf dem weiteren Bildungsweg trainieren Schüler und Studenten das Verfassen kohärenter 
wissenschaftlicher Texte und die damit verbundenen Fähigkeiten. Dabei stellt sowohl die nötige Recherche als auch das Schreiben selbst
oft eine Herausforderung dar. Wie künstliche Intelligenz eingesetzt werden kann, um Schüler und Studenten bei diesen Aufgaben zu 
unterstützen, soll in den folgenden Abschnitten erläutert werden. 

\section{Self-Efficacy}

Eine Anfang 2025 veröffentlichte Studie von Yulu Cui untersucht die Gründe, weshalb sich Studenten für die Nutzung von 
künstlicher Intelligenz beim Verfassen akademischer Arbeiten entscheiden. Dabei wird der Fokus vor allem auf emotionale 
Aspekte gelegt. Besonders personalisiertes Feedback und eine einfach zu bedienende Nutzeroberfläche seien entscheidend.\cite{SelfEfficacyBeliefs} \\
Dabei verweist Cui mehrfach auf das Konzept Self-Efficacy. Selbstwirksamkeit beschreibt in diesem Zusammenhang den Glauben 
an die eigene Fähigkeit, eine Aufgabe erfolgreich zu erledigen. Schätzt ein Student seinen Arbeitsstil als effizient ein, 
verfügt er auch über ein stärkeres Gefühl der Selbstwirksamkeit.\\
Nach van Blankenstein et al. stellt die Aufgabe des wissenschaftlichen Schreibens besonders für noch unerfahrene Studenten 
ein große Herausforderung dar. Bereits 1999 stellten Pajares, Miller, and Johnson sowie Pajares and Valiante fest,
dass die Befürchtung, beim Schreiben zu versagen oder schlechte Erfahrungen zu machen, sich negativ auf die Schreibleistung 
auswirken.\cite{writingSelfBeliefs,writingSelfBeliefsMiddleSchool}

Das Gefühl, während des Schreibprozesses über eine hohe Selbstwirksamkeit (Self-Efficacy) zu verfügen, vereinfache den 
Schreibprozess und führe somit auch zu besseren Ergebnissen.\cite{SelfEfficacyBeliefs} \\
Durch den richtigen Einsatz von KI während verschiedener Schritte des wissenschaftlichen Schreibprozesses erhält der Student
ein Gefühl besserer Effizienz beim arbeiten, was sich positiv auf das Gefühl der Selbstwirksamkeit und damit auf den Erfolg
beim Schreiben auswirkt.\\
KI kann beispielsweise sehr schnell Feedback zu geschriebenen Textteilen geben. Dies helfe den Studenten, ihre eigene 
Leistung besser einzuschätzen und besser mit negativen Emotionen umzugehen beziehungsweise diese ganz zu vermeiden. Somit 
können sie sich leichter wieder auf die eigentliche Aufgabe konzentrieren.\cite{SelfEfficacyBeliefs} 



\section{Vor- und Nachteile der Nutzung von KI}
\section{bereits bestehende Tools}


\end{document}