\documentclass[../main.tex]{subfiles}
\begin{document}

\section{Bedeutung des Wissenschaftlichen Schreibens}
\label{sec:bedeutung}
Mehrere Studien betonen die Bedeutung des wissenschaftlichen Schreibens für die persönliche und akademische Entwicklung 
von Schülern und vor allem \mbox{Studenten \cite{influencingUsingAi,ZukunftWissenschaftlichesPublizieren}.} Dabei geht es ebenso um das Erlernen der Fähigkeit, sich verständlich auszudrücken und 
eigene Ideen und Gedanken strukturiert vermitteln zu können, wie auch um die Kompetenz, sich durch Recherche neues Wissen 
anzueignen, dieses zu strukturieren und in andere Zusammenhänge zu bringen.
%TODO: sehr lang der satz
Zudem sind Organisation und Planung wichtige Fähigkeiten, die zu dem erfolgreichen Verfassen eines wissenschaftlichen Textes benötigt werden.\cite{SelfEfficacyBeliefs} \\
Für den entstehenden Text sind sowohl der Inhalt als auch die Struktur relevant. Der Inhalt bildet den Kern des Textes. Er umfasst die Ideen und Gedanken des Autors und sollte 
informativ, bedeutsam und originell sein. Der Inhalt sorgt dafür, dass der geschriebene Text seinen Zweck erfüllt, beispielsweise 
den Leser über ein \mbox{Thema zu informieren.} \\
Die Struktur des Textes hingegen beinhaltet die Art des Schreibens, die Anordnung des Inhalts und das Herstellen eines Leseflusses. 
Ein gut strukturierter Text gibt Ideen in zusammenhängender und logischer Reihenfolge wieder. Eine gute Struktur hilft, den Inhalt des
Textes zu vermitteln. Dementsprechend sollte bei dem Prozess des wissenschaftlichen Schreibens der Inhalt wie auch die Struktur des entstehenden 
Textes beachtet werden. \cite{teachers}\\
Sowohl während der Schulzeit als auch auf dem weiteren Bildungsweg trainieren Schüler und Studenten das Verfassen kohärenter 
wissenschaftlicher Texte und die damit verbundenen Fähigkeiten. Dabei stellt die nötige Recherche ebenso wie das Schreiben selbst
häufig eine Herausforderung dar. Wie \glslink{glos:ki}{KI} eingesetzt werden kann, um Schüler und Studenten bei diesen Aufgaben zu 
unterstützen, soll in den folgenden Abschnitten erläutert werden. 

\section{Selbstwirksamkeit während des Schreibprozesses}
\label{sec:efficacy}

Eine Anfang 2025 veröffentlichte Studie von Yulu Cui untersucht die Gründe, weshalb sich Studenten für die Nutzung von 
%TODO: macht hier eine in-text citation oder lasst den namen weg
\glslink{glos:ki}{KI} bei dem Verfassen akademischer Arbeiten entscheiden. Der Fokus wird vorrangig auf emotionale 
Aspekte gelegt. Besonders personalisiertes Feedback und eine intuitiv zu bedienende Nutzeroberfläche \mbox{seien entscheidend.\cite{influencingUsingAi}} \\
Dabei verweist Cui mehrfach auf das Konzept Self-Efficacy (dt.: Selbstwirksamkeit). Selbstwirksamkeit beschreibt in diesem Zusammenhang den Glauben 
an die eigene Fähigkeit, eine Aufgabe erfolgreich zu erledigen. Schätzt ein Student seinen Arbeitsstil als effizient ein, 
verfügt er über ein stärkeres Gefühl der Selbstwirksamkeit. \cite{influencingUsingAi,SelfEfficacyBeliefs}\\
Nach van Blankenstein et al. stellt die Aufgabe des wissenschaftlichen Schreibens insbesondere für noch unerfahrene Studenten 
%TODO: auch solltet ihr eine richtige citation nutzen können
ein große Herausforderung dar. Bereits 1999 stellten Pajares, Miller und Johnson sowie Pajares und Valiante fest,
dass die Befürchtung, während des Schreibens zu versagen oder schlechte Erfahrungen zu machen, sich negativ auf die Schreibleistung 
auswirken. \cite{writingSelfBeliefs,writingSelfBeliefsMiddleSchool}\\
Das Gefühl, während des Schreibprozesses über eine hohe Selbstwirksamkeit zu verfügen, vereinfache den 
Schreibprozess und führe somit zu besseren Ergebnissen. \cite{SelfEfficacyBeliefs} \\
Durch den richtigen Einsatz von \glslink{glos:ki}{KI} während verschiedener Schritte des wissenschaftlichen Schreibprozesses erhält der Student
ein Gefühl besserer Effizienz während des Arbeitens, was sich positiv auf das Gefühl der Selbstwirksamkeit und damit auf den Erfolg
beim Schreiben auswirkt.\\
%TODO: gibt es hierfür auch eine quelle? wenn ja, super, wenn nein dann fourmuliert das also hypothese oder so
\glslink{glos:ki}{KI} kann beispielsweise schnell Feedback zu geschriebenen Textteilen geben. Dies helfe den Studenten, ihre eigene 
Leistung besser einzuschätzen und besser mit negativen Emotionen bezüglich ihrer Fähigkeiten im Schreibprozess umzugehen, beziehungsweise diese ganz zu vermeiden. Somit 
können sie sich leichter wieder auf die eigentliche Aufgabe konzentrieren. \cite{SelfEfficacyBeliefs} 



\section{Vor- und Nachteile der Nutzung von Künstlicher Intelligenz im Schreibprozess}

\subsection{Einsatzmöglichkeiten und Vorteile}
\label{sec:vorteile}

Nicht nur für schnelles Feedback zu bereits geschriebenen Texten kann \glslink{glos:ki}{KI} während des Schreibprozesses eingesetzt werden. 
%TODO: das "nicht nur" klingt für mich am anfang hier ein bisschen komoisch
Weitere Verwendungsmöglichkeiten sind das Umformulieren von Texten, das Zusammenfassen von Literatur, das Formulieren von Texten aus Stichpunkten und die Verbesserung des Stils sowie der Rechtschreibung und Grammatik von bereits geschriebenen Texten. Diese Einsatzmöglichkeiten 
tragen dazu bei, die Leistungsfähigkeit während des Schreibprozesses zu erhöhen und somit das Gefühl der Selbstwirksamkeit des Autors zu steigern. \cite{SelfEfficacyBeliefs}\\
%TODO: steht diese schluss das ki das ermöglicht auch in dem paper oder ist das euere eigene schlussfolgerung? wenn es euere eigene ist würde ich das so formulieren dass das klar ist
Die Erstellung einer Kurzfassung, welche für jede wissenschaftliche Arbeit notwendig ist, jedoch keine geistige Schöpfungshöhe vom Autor mehr verlangt, 
%TODO: "keine geistige Schöpfungshöhe" ist wohl ein bisschen weit gegriffen, oder?
kann von \glslink{glos:ki}{KI} übernommen werden. Ebenso eignet sie sich zur Korrektur von Texten. \glslink{glos:ki}{KI}-Werkzeuge können komplexe Sprache vereinfachen, falsch 
verwendete Wörter ersetzen und durch das Vorschlagen passender Fachbegriffe und Synonyme die sprachliche Vielfalt eines Textes erhöhen. Durch eine 
automatische Korrektur von Grammatik und Rechtschreibung, sowie Vorschläge zum Umstellen der Satzstrukur, lässt sich der Lesefluss eines Textes verbessern.\cite{ZukunftWissenschaftlichesPublizieren,teachers}\\
Eine weitere Verwendungsmöglichkeit für \glslink{glos:ki}{KI} ist das Generieren von Forschungsfragen und die Unterstützung bei der Suche nach möglichen Forschungsthemen. Da \glslink{glos:ki}{KI}-Modelle mit einer großen Wissensbasis trainiert werden, 
können so neue und interdisziplinäre Ideen entstehen.\cite{ZukunftWissenschaftlichesPublizieren,humanWritingToAi}\\
Eine im Jahr 2023 veröffentlichte Studie zu der Nutzung von \glslink{glos:ki}{KI}-Schreibwerkzeugen von indonesischen Lehrkräften stellt fest, dass sich die Qualität der von 
Schülern geschriebenen Texte durch die Nutzung von \glslink{glos:ki}{KI} erhöht. Die Texte seien klarer formuliert, enthielten weniger Fehler und 
wirkten allgemein kohärenter. Zudem unterstützten die \glslink{glos:ki}{KI}-Werkzeuge die Schüler, Schreibblockaden zu überkommen und Ideen für das Schreiben von Texten zu entwickeln.\cite{teachers} 

%TODO: dem abschnitt fehlt ein bisschen ein roter faden: gestaltet das ganze vielleicht als list?

\subsection{Risiken und Nachteile}
\label{sec:nachteile}

Trotz der genannten Vorteile und Einsatzmöglichkeiten gibt es immer wieder Kritik an der Verwendung von \glslink{glos:ki}{KI} während des Verfassens wissenschaftlicher Texte. 
Die Nutzung von \glslink{glos:ki}{KI}-Werkzeugen während des wissenschaftlichen Schreibens verlangt von Studenten vor allem die Kompetenz, generierte Inhalte kritisch zu 
hinterfragen. Mehrere Studien äußern Bedenken, dass die Fähigkeit des kritischen Denkens, sowie Kreativität und Originalität des Autors verloren gehen könnten. \cite{ZukunftWissenschaftlichesPublizieren,teachers,BucherSchwarzerHolzwweißig}\\
Überdies ergibt sich durch die in \autoref{sec:llm} beschriebene Funktionsweise von \glslink{glos:ki}{KI} eine Plagiatsgefahr. \glslink{glos:ki}{KI}-Werkzeuge wie \glslink{chatgpt}{ChatGPT} generieren einen Text Wort für Wort, wobei das nächste anhand 
einer Liste von Wahrscheinlichkeiten ausgewählt wird. Dabei passiert es häufig, dass ein Text aus den Trainingsdaten exakt oder leicht verändert wiedergegeben wird, wie die New York Times in 
einer Klage gegen OpenAI nachweist \cite{NYTimes}. Es besteht also die Möglichkeit, durch die Verwendung von \glslink{glos:ki}{KI} unbeabsichtigt Plagiate zu erstellen. Der deutsche Hochschulverband definiert Plagiate als 
die "`wörtliche und gedankliche Übernahme fremden geistigen Eigentums ohne entsprechende Kenntlichmachung"' \cite{Hochschulverband}. \\ Bei der Nutzung von \glslink{glos:ki}{KI}-Werkzeugen fehlt es häufig an Transparenz, 
%TODO: diese definition kommt hier ein bisschen aus dem nichts
zu welchen Anteilen der entstandene Text aus neu generierten Inhalten besteht oder lediglich die Kopie von Trainingstexten ist. Daraus ergibt sich die Notwendigkeit, \glslink{glos:ki}{KI}-generierte Texte nicht nur 
auf den Inhalt, sonder darüber hinaus auf potentielle Plagiate zu prüfen. Dazu gibt es unter anderem \glslink{glos:ki}{KI}-Plagiatsdetektoren, welche jedoch nicht immer zuverlässig sind. Besonders wenn ein Text 
leicht verändert wiedergegeben wird, ist dies schwer zu erkennen.\\ Plagiatsvorwürfe können "`zum Nichtbestehen von Prüfungsleistungen, Aberkennung von Abschlüssen  oder  zur  Zwangsexmatrikulation  
führen"'\cite{Plagiate}. Aufgrund dieser potentiellen Folgen liegt es im Interesse von Studenten, Plagiatsvorwürfe zu vermeiden. Viele Hochschulen sehen bereits die generelle Verwendung von \glslink{glos:ki}{KI} bei Hausarbeiten 
oder ähnlichen Leistungen als Plagiat an, da ein mit \glslink{glos:ki}{KI} generierter Text keine eigene Schöpfungshöhe aufweist, aber trotzdem als eigene Leistung ausgegeben wird. Andere erlauben die Verwendung von 
\glslink{glos:ki}{KI} mit entsprechender Kennzeichnung. \cite{Plagiate}\\
Zudem kann der übermäßige Einsatz von \glslink{glos:ki}{KI} dazu führen, dass Schüler und Studenten das Verständnis für die eigenen Texte fehlt. Schüler könnten die im 
Kontext verwendeten, von der \glslink{glos:ki}{KI} vorgeschlagenen, Fachbegriffe beispielsweise nicht mehr verstehen. Sollten sie Synonyme, Fachbegriffe und Formulierungen 
übernehmen, ohne diese zu hinterfragen, kann dies zu einem übertrieben förmlichen und schwer verständlichen Schreibstil führen \cite{teachers}. 
So kann sowohl die Kreativität als auch der persönliche Schreibstil verloren gehen. Da \glslink{glos:ki}{KI}-Modelle auf Grundlage verschiedener Texte trainiert werden, ist es
schwer, während der Verwendung eines \glslink{glos:ki}{KI}-Schreibwerkzeugs zum Schreiben oder Umformulieren von Texten einen einheitlichen und persönlichen Schreibstil aufrechtzuerhalten\cite{creativeWriting}. Hinzu kommt,
dass \glslink{glos:ki}{KI}-Modelle aufgrund ihrer in \autoref{sec:llm} erklärten Funktionsweise Schwerpunkte nicht nach inhaltlichen, sondern mathematischen Kriterien legen \cite{berensBolk}. Somit kann der 
Einsatz von \glslink{glos:ki}{KI} dem Inhalt ebenso wie der Struktur eines Textes schaden, welche, wie in \autoref{sec:bedeutung} beschrieben, Relevanz für die Qualität des Textes haben.\\
Ein Artikel des Open-Access-Publikationsportals "`German Medical Science"' zu dem Thema "`Künstliche Intelligenz und ChatGPT: Über die Zukunft des wissenschaftlichen Publizierens"'
betont insbesondere die Eigenleistung, welche zum Verfassen einer wissenschaftlichen Arbeit erforderlich ist. Die \glslink{glos:ki}{KI} solle lediglich als Assistent dienen, und
%TODO: Der genaue titel name und veröffentlicher ist total unrelevant hier. cited einfach die quelle und macht eueren punkt
nicht die Rolle des Verfassers annehmen, dementsprechend keinen eigenen Inhalt generieren. \cite{ZukunftWissenschaftlichesPublizieren} \\ Sollte ein \glslink{glos:ki}{KI}-Modell verwendet werden, welches keinen Internetzugang hat,
fehlem diesen aktuelle Daten und Forschungsergebnisse. Wird ein nicht lokal laufendes Modell verwendet, kann durch die Verarbeitung der Daten durch den \glslink{glos:ki}{KI} Anbieter
der Datenschutz nicht mehr gewährleistet werden.\cite{ZukunftWissenschaftlichesPublizieren} \\ Bucher, Holzweißig und Schwarzer betonen vor allem die Gefahr, dass 
generative \glslink{glos:ki}{KI} halluzinieren und damit Falschinformationen hervorbringen kann. So können wissenschaftliche Fakten durch \glslink{glos:ki}{KI} verzerrt werden. In ihrem Buch "`Künstliche Intelligenz und wissenschaftliches Arbeiten"' nennen sie drei 
%TODO: same point wie oben, ich würde nicht den ganzen titel hier hinschreiben, interessiert keinen an dieser stelle
Gründe, welche gegen die Verwendung von \glslink{glos:ki}{KI} im Studium sprechen: rechtliche sowie ethische Bedenken, der Einsatz von \glslink{glos:ki}{KI} wird in der Prüfungsordnung untersagt, 
der Einsatz von \glslink{glos:ki}{KI} wird vom Partnerunternehmen im Zuge eines dualen Studiums, primär aus datenschutzrechtlichen Gründen, untersagt.\cite{BucherSchwarzerHolzwweißig} \\Auch einige Verläge lassen 
die Nutzung von \glslink{glos:ki}{KI} zu dem Erstellen wissenschaftlicher Publikationen nicht zu, oder fordern zu mindest eine Kennzeichnung\cite{ZukunftWissenschaftlichesPublizieren}.\\ 
Häufig wird eine Art "`\glslink{augmented-intelligence}{Augmented Intelligence}"' gefordert, also ein Hilfsmittel, welches zwar die Effizienz steigert, aber selber nicht das Verfassen der 
Arbeit und die damit verbundene Verantwortung übernimmt. Schüler und Studenten sollen die generierten Inhalte kritisch hinterfragen und Fakten überprüfen. Die Kreativität und 
Eigenverantwortung des Autors \mbox{soll erhalten bleiben. \cite{BucherSchwarzerHolzwweißig,humanWritingToAi,teachers,ZukunftWissenschaftlichesPublizieren}} 

\section{Überblick über bestehende \glslink{glos:ki}{KI}-Lösungen}
\label{sec:bereitsBestehendeLoesungen}

Obwohl \glslink{glos:llm}{LLM}s erst seit kurzer Zeit ihre aktuelle Popularität erreicht haben, gibt es schon einige Modelle und Tools, welche unter anderem beim Schreiben \mbox{unterstützen können.} \\
So gibt es zum Beispiel zahlreiche Online-\glslink{glos:ki}{KI}-Chats, angefangen bei den OpenAI-Modellen wie ChatGPT. Werden diese bei dem Verfassen wissenschaftlicher Texte zur Hilfe genommen, ergeben 
%TODO: ChatGPT ist kein KI-Modell sondern ein Produkt von OpenAI, welches auf unterschiedlichen KI-Modellen basiert
sich jedoch einige Probleme. Wie bei allen nicht lokal laufenden Modellen ergibt sich ein Datenschutzproblem. Im Übrigen fallen für die gestellten Anfragen Kosten für das Betreiben der 
externen Server an, welche dann dem Nutzer berechnet werden. Mit lokal laufenden Modellen lassen sich diese Probleme umgehen. Außerdem sind diese Modelle meist nicht spezialisiert, 
können keine Chats speichern und die Funktionalität kann im Rahmen einer benoteten Arbeit nicht eingeschränkt werden. So könnten Schüler bei der Nutzung einen gesamten Aufsatz von 
der \glslink{glos:ki}{KI} generieren lassen, ohne eigene Leistung zu erbringen. OpenAI bietet mit "`SchulKI"' ein auf Schulen spezialisiertes Produkt an. Trotz dieser Spezialisierung bleiben die 
Schwierigkeiten eines Online-Tools bestehen.\cite{schulki}\\ 
Um mit Online-Modellen verbundene Nachteile, wie die Abhängigkeit von in Schulen möglicherweise schlechtem WLAN sowie die beschriebenen Datenschutzprobleme zu umgehen, bieten sich 
lokal laufende Modelle an. Die \glslink{open-source}{Open-Source}-Software \glslink{ollama}{Ollama} stellt zur Nutzung solcher Modelle eine \acrshort{api1} zur Verfügung, stellt jedoch selber keine grafische 
Benutzeroberfläche bereit, sondern wird über das Terminal verwendet. Gerade für Schüler könnte dies eine Herausforderung darstellen. Zudem sind diese Modelle nicht auf Schreibaufgaben 
spezialisiert. Darüber hinaus ergeben sich Nachteile in der Performanz, da die Verwendung von der Rechenleistung des Nutzergerätes abhängig ist.\cite{ollamaSchreibassi}\\
Ein weiteres auf den Einsatz in Schulen spezialisiertes Produkt ist der Fobizz Schulassistent. Dieser soll Lehrer während des Unterrichtens unterstützen, indem er zum Beispiel 
Matheaufgaben in anderen Worten erklärt. Dabei ist dieses Werkzeug eher ein Assistent für Lehrer sowie ein Lernassistent, und erfüllt nicht die Funktionalität eines spezialisierten 
Schreibassistenten.\cite{fobizz}\\ 
Werkzeuge wie die automatische Rechtschreibungs- und Grammatikprüfung von Microsoft Word oder das \glslink{glos:ki}{KI}-gestützte Textüberprüfungswerkzeug Grammarly bieten bereits eine gute 
Unterstützung während des Schreibprozesses. Jedoch lassen sich keine Chats mit \glslink{glos:ki}{KI}-Modellen erstellen. Die Funktionalität beschränkt sich ausschließlich auf die Korrektur von 
Rechtschreibung und Grammatik, sowie im Fall von Grammarly einzelne Vorschläge für das Umformulieren von Texten.\cite{microsoftword,grammarly}\\
Es gibt also bereits einige, teilweise auf den Schreibprozess spezialisierte, \glslink{glos:ki}{KI}-Lösungen. Jedoch verfügt keines der beschriebenen Hilfsmittel über den vollen in \autoref{sec:vorteile}
definierten Funktionsumfang und eine auf die Bedürfnisse von Schülern und Studenten angepasste Benutzeroberfläche.\\
Auch die Nachvollziehbarkeit, wie viel \glslink{glos:ki}{KI} in einem Text verwendet 
wurde, geht bei der Nutzung häufig verloren. Im schulischen und universitären Kontext besteht also die Gefahr, dass ganze Abgaben ungekennzeichnet mit \glslink{glos:ki}{KI} verfasst werden. Sollte ein 
Schreibassistent in der Schule während des Verfassens benoteter Arbeiten eingesetzt werden, sollten benotende Lehrer beispielsweise durch ein \glslink{glos:ki}{KI}-Nutzungsverzeichnis 
nachvollziehen können, in welchem Umfang \glslink{glos:ki}{KI} \mbox{verwendet wurde.}


\end{document}
