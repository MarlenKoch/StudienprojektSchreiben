\documentclass[../main.tex]{subfiles}
\begin{document}


\section{Evaluation}
Das Ergebnis dieses Projektes ist ein Schreibassistent, welcher eine Nutzeroberfläche für \glslink{ollama}{Ollama} bietet.
Zwischen den \glslink{glos:ki}{KI}-Modellen kann schnell gewechselt werden. Darüber hinaus kann der Nutzer Chats speichern und ein \glslink{glos:ki}{KI}-Nutzungsverzeichnis generieren lassen. Durch das Hinzufügen von Kommentaren 
zu \glslink{glos:ki}{KI}-Antworten kann der Nutzer eigene Gedanken zu den generierten Inhalten an einem Ort sammeln. Durch diese Funktionalitäten wird der Einsatz von \glslink{glos:ki}{KI} im Schreibprozess, im Vergleich 
zu der Nutzung mehrerer unabhängiger \glslink{glos:ki}{KI}-Chats, in welchen keine Informationen gespeichert werden können, erleichtert. Zudem muss für die Handhabung lokaler \glslink{glos:ki}{KI}-Modelle kein Terminal genutzt werden. \\ 
Um die Funktionalität des Schreibassistenten zu evaluieren, wurde dieser zur stichprobenartigen Umformulierung einiger Absätze zur Hilfe genommen (\autoref{sec:erklärbarkeitsproblem}, \autoref{ch:bewertung}). Aufgrund der Parallelisierung des Schreibens der Arbeit und der Entwicklung des Schreibassistenten, welche es ermöglicht hat, die Ergebnisse der Literaturquellen bei der 
Implementierung zu berücksichtigen, war es nicht möglich, die gesamte Arbeit mithilfe des Schreibassistenten zu erstellen. \\
Um einen größtmöglichen Mehrwert zu bieten, orientiert sich das Design des Schreibassistenten an den Ergebnissen der in \autoref{ch:anforderungen} aufgeführten Studien. So wurden beispielsweise 
\glslink{tooltips}{Tooltips} hinzugefügt, um das Verständnis für die Funktionalität zu erleichtern. Dadurch soll die Nutzung von \glslink{glos:ki}{KI} für Menschen vereinfacht werden, welche bisher noch keine Erfahrung im 
Umgang mit dieser Technologie haben.\\
Die tatsächliche Effizienzsteigerung lässt sich nicht empirisch nachweisen, da aufgrund des zeitlichen Umfangs des Projektes noch keine Nutzerbefragungen oder ähnliche Untersuchungen durchgeführt 
wurden. Als nächsten Schritt könnten genauere Untersuchungen durchführen werden. Dazu könnten Schüler und Studenten befragt, oder die Qualität mit und ohne 
Einsatz des Schreibassistenten verfasster Texte verglichen werden.\\ 
Außerdem wurden die im Schreibassistenten empfohlenen \glslink{glos:ki}{KI}-Modelle lediglich stichprobenartig getestet und nicht anhand wissenschaftlicher Kriterien bewertet. Es ist möglich, dass andere Modelle
bessere Ergebnisse liefern.\\
Insgesamt zeigt sich, dass der Schreibassistent bereits zahlreiche Vorteile für den Schreibprozess bietet, jedoch noch Potenzial für weiterführende Verbesserungen und Erweiterungen 
besteht. Im Folgenden sollen daher die aktuellen Grenzen des Systems sowie mögliche Ansätze zur Weiterentwicklung diskutiert werden.\\
Die Nutzung des entwickelten \glslink{glos:ki}{KI}-Werkzeugs ist von der zur Verfügung stehenden Technik abhängig. Durch die Verwendung lokal auf dem Rechner des Nutzers laufender Modelle werden zwar 
Datenschutzprobleme und bei einem \glslink{glos:ki}{KI}-Betreiber anfallende Nutzungskosten vermieden, jedoch ist die Performanz der \glslink{glos:ki}{KI}-Modelle von der Rechenleistung des verwendeten Geräts abhängig. 


\section{Ausblick}
Um dieses Problem zu lösen, sollte in Zukunft das Einbinden externer \glslink{glos:ki}{KI}-Modelle ermöglicht werden. Um dennoch den Datenschutz zu gewährleisten, könnte ein weiteres \glslink{glos:llm}{LLM}
eingebunden werden, welches alle persönlichen Daten aus den Nutzerangaben herausfiltert und vor der Weitergabe an ein externes Modell zensiert.\\ 
Darüber hinaus sollte vor allem eine effiziente und sichere Nutzerverwaltung implementiert werden. Für die Verwendung des Schreibassistenten in Schulen sollten Lehrern Admin-Accounts 
zur Verfügung stehen, denen sich mehrere Schüler-Accounts zuordnen lassen. Der Lehrer kann so beispielsweise die Bearbeitungszeit während des Prüfungsmodus einstellen. Außerdem 
würden abgegebene Dateien der Schüler im Lehrer-Account angezeigt werden.\\
Für Studenten könnte durch das Nutzermanagement ein kollaborativer Modus hinzugefügt werden, welcher das Erstellen von Arbeiten mit mehreren \mbox{Autoren vereinfacht.}\\
Des Weiteren könnte die Funktionalität zu einem vollständigen Texteditor erweitert werden, sodass die bisher genutzten Editoren nicht nur ergänzt, sonder durch den Schreibassistenten 
abgelöst werden. Dazu wäre das Ermöglichen einer flexibleren Formatierung des geschriebenen Textes, beispielsweise mit Unterüberschriften, nötig. Es sollte möglich sein, ein 
Literaturverzeichnis mit Verweisen hinzuzufügen. Ebenfalls sollte ein Deckblatt und Inhalts- wie auch Abbildungs- und Abkürzungsverzeichnis eingefügt werden können.

\end{document}