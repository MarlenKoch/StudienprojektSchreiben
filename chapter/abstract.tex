\documentclass[../main.tex]{subfiles}
\begin{document}

Wissenschaftliches Schreiben stellt eine grundlegende Kompetenz für Schüler und Studierende 
dar und umfasst neben dem Verfassen von Texten auch die Organisation und Strukturierung von Wissen. 
Mit dem zunehmenden Einzug digitaler Medien, insbesondere KI-gestützter Werkzeuge wie ChatGPT, verändern 
sich die Anforderungen und Möglichkeiten im Schreibprozess. Der Einsatz künstlicher Intelligenz bietet sowohl 
Chancen zur Unterstützung als auch neue Herausforderungen.\\
Diese Arbeit diskutiert die Potenziale und Risiken von KI im wissenschaftlichen Schreiben und beleuchtet den 
Prozess der Erstellung eines innovativen Schreibassistenten, der die Nutzung von KI erleichtern und sicherer gestalten soll.
 Ziel ist es, den Schreibprozess für Lernende effizienter und zugänglicher zu machen.

\end{document}
