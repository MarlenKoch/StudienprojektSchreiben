\documentclass[../main.tex]{subfiles}
\begin{document}

Der Einsatz Künstlicher Intelligenz (KI), insbesondere Large Language Models (LLMs), verändert den wissenschaftlichen Schreibprozess von Schülern und Studierenden grundlegend. \\
Die vorliegende Arbeit diskutiert Chancen und Herausforderungen des Einsatzes von KI im wissenschaftlichen Schreiben und untersucht unter Berücksichtigung der Bedürfnisse von Schülern und Student, wie sich KI-gestützte Werkzeuge auf die Selbstwirksamkeit und Kompetenzentwicklung von Lernenden auswirken.

Basierend auf den identifizierten Anforderungen wird die Implementierung eines KI-Schreibassistenten vorgestellt, der gezielt für den schulischen und universitären Kontext entwickelt wurde. Das System setzt auf lokal betriebene LLMs, um Datenschutz zu gewährleisten, und bietet eine intuitive Benutzeroberfläche, verschiedene Nutzungsmodi  sowie Maßnahmen zur Nachvollziehbarkeit und Risikominderung des KI-Einsatzes. \\
Ziel des KI-Schreibassistenten ist es, Effizienz und Qualität im Schreibprozess zu steigern und zugleich einen kritischen und verantwortungsvollen Umgang mit KI zu fördern.

\end{document}
