\documentclass[../main.tex]{subfiles}
\begin{document}

Mit einem Aufschwung der KI-Technologien findet sich diese in vielen Anwendungsfällen wieder.
Dies gilt auch für die Software-Entwicklung.
Dieser Bericht untersucht, wie \glspl{LLM} zur Durchsuchung von Programmcode eingesetzt werden können, um einen Überblick über Projekte zu gewinnen und zu bewahren sowie Entwicklungsprozesse zu optimieren.
Dafür wird der Ansatz verfolgt, einen Index aus sogenannten Vektor Embeddings zu erstellen, welcher dann mit Fragen natürlicher Sprache durchsucht werden kann.
Für das Erstellen der Vektor Embeddings werden \glspl{LLM} genutzt.
Es wird ein Testprojekt und Testfragen herangezogen, um die im folgenden Bericht erarbeiteten Verfahren dafür zu evaluieren.

\end{document}