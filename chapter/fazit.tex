\documentclass[../main.tex]{subfiles}
\begin{document}

Der im Rahmen dieses Studienprojektes implementierte Schreibassistent erfüllt die in neueren Studien identifizierten Anforderungen an ein während des Schreibprozesses eingesetztes 
\glslink{glos:ki}{KI}-Hilfsmittel. Damit wird das Ziel verfolgt, den Schreibprozess effizienter zu gestalten und somit das Gefühl der Selbstwirksamkeit von Schülern und Studenten während des 
Schreibprozesses zu stärken. Dies fördert ebenfalls das Vertrauen der Schüler und Studenten in ihre eigenen Fähigkeiten. Der Einsatz des \glslink{glos:ki}{KI}-Assistenten fördert sprachlich vielfältige 
und grammatikalisch korrekte Texte, und verbessert den Lesefluss. \\
Durch das Design des \glslink{glos:ki}{KI}-Werkzeugs sollen die Risiken der \glslink{glos:ki}{KI}-Nutzung minimiert werden. Dazu wurden \glslink{tooltips}{Tooltips} eingerichtet, welche das Verständnis des Nutzers für die Funktionsweise des 
Schreibassistenten fördern sollen. Darüber hinaus wird der Nutzer vor \glslink{glos:ki}{KI}-generierten Falschinformationen gewarnt, was sich in einer Studie von Hwang und Jeong (2025) für sinnvoll 
erwiesen hat. \\
Die unterschiedlichen Modi des Schreibassistenten ermöglichen einen vielfältigen Einsatz. Damit dies jedoch auch im schulischen Kontext erfolgreich geschehen kann, muss noch eine 
effiziente Nutzerverwaltung implementiert werden, sodass die Schüler-Accounts von einem Admin-Account beaufsichtigt werden können.\\
Darüber hinaus würde eine Erweiterung der Funktionalität die Nutzererfahrung noch weiter verbessern, sodass wissenschaftliche Arbeiten im universitären Kontext ausschließlich 
mit dem Schreibassistenten verfasst werden können. Dazu wäre unter anderem eine bessere Textformatierung nötig. Damit könnte der Schreibassistent nicht nur unterschiedliche \glslink{glos:ki}{KI}-Chats vereinen und übersichtlicher gestalten, sondern bisher genutzte 
Editoren ablösen.\\
In Schulen kann der Einsatz des \glslink{glos:ki}{KI}-Assistenten den Umgang mit modernen Technologien fördern. Die Schüler lernen so in einem kontrollierten Umfeld, mit generativer \glslink{glos:ki}{KI} umzugehen. 
Dies ermöglicht ihnen, diese in Zukunft kritischer zu hinterfragen. Zudem bringt ihnen diese Erfahrung aufgrund der Popularität generativer \glslink{glos:ki}{KI} in Unternehmen Vorteile im Berufsleben.\\
Abschließend lässt sich festhalten, dass der entwickelte Schreibassistent eine Grundlage für die Integration von KI in schulische und universitäre Schreibprozesse bietet. 
Zentrale Anforderungen wie Benutzerfreundlichkeit, Transparenz und die Minimierung von Risiken wurden den analysierten Studien entsprechend umgesetzt. 
Damit können Effizienz und Qualität im wissenschaftlichen Arbeiten gesteigert und notwendige Medienkompetenzen vermittelt werden. Die Umsetzung der genannten Erweiterungsmöglichkeiten würde 
den Mehrwert weiter steigern. 

\end{document}
 