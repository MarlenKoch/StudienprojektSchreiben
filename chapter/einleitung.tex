\documentclass[../main.tex]{subfiles}
\begin{document}

Wissenschaftliches Schreiben ist ein essenzieller Teil der schulischen und akademischen Ausbildung. Die Relevanz, in Schulen Wert auf die Entwicklung der Schreibfertigkeiten von Schülern zu legen, 
betonte bereits 2003 die National Commission on Writing in America’s Schools \& Colleges \cite{nationalcommissionwriting}.\\
Der Prozess des wissenschaftlichen Schreibens stellt diverse 
Anforderungen an einen Schüler oder Studenten. Die Fähigkeit, sich selbst zu organisieren und eigenes oder durch Recherche erlangtes Wissen zu strukturieren, ist essenziell. Durch das 
Schreiben wissenschaftlicher Arbeiten trainieren Schüler und Studenten ihre Fähigkeit, Ideen und Konzepte verständlich auszuformulieren. Dies fördert ebenfalls die Fähigkeit, klar zu 
kommunizieren, wodurch bessere Zusammenarbeit im Team gewährleistet wird. \cite{nationalcommissionwriting,teachers,humanWritingToAi} \\ 
Während der Recherche zu wissenschaftlichen Themen wird zudem die Medienkompetenz gestärkt. Zu den häufig verwendeten Medien zählen heute nicht nur Suchmaschinen wie Google Scholar, sondern 
auch diverse Anwendungen, die \gls{ki} nutzen.\\ Nutzen Schüler und Studenten KI-gestützte Hilfsmittel, sammeln sie Erfahrungen, die später im Berufsleben hilfreich sein können. Laut 
einer Mitteilung des statistischen Bundesamtes aus dem Jahr 2024 nutzt bereits jedes fünfte Unternehmen KI, und der Anteil wird voraussichtlich weiter steigen \cite{statistischesBundesamt}. 
Die Technologie wird zur Analyse und Erzeugung von natürlicher Sprache eingesetzt \cite{statistischesBundesamt}.\\ 
Spätestens seit der Veröffentlichung von \glslink{chatgpt}{ChatGPT} durch OpenAI im Jahr 2022, einem KI-Chatbot, welcher Texte generieren kann, die menschengeschriebenen ähnlich sind, wird die Verwendung von 
\glslink{glos:ki}{KI} im wissenschaftlichen Schreiben viel diskutiert \mbox{und untersucht. \cite{humanWritingToAi,ZukunftWissenschaftlichesPublizieren}}\\
Durch den Einsatz von \glslink{glos:ki}{KI} im Schreibprozess werden Vor- und Nachteile deutlich. Diese Arbeit befasst sich mit 
diesen Aspekten sowie mit möglichen Lösungsvorschlägen zum Umgang mit den Nachteilen. Insbesondere wird ein neues Schreibassistenten-Programm vorgestellt, das 
die Risiken von \glslink{glos:ki}{KI} minimieren und den Einsatz im Schreibprozess vereinfachen soll. Dazu werden zunächste mehrere wissenschaftliche Untersuchungen beleuchtet und anschließend die erarbeiteten Anforderungen an ein KI-Schreibwerkzeug
umgesetzt und die Ergebnisse bewertet.

\end{document}
