\documentclass[../main.tex]{subfiles}
\begin{document}

Wissenschaftliches Schreiben ist ein essenzieller Teil des Schüler- und Studentendaseins. Die Relevanz, in Schulen Wert auf die Entwicklung der Schreibfertigkeiten von Schülern zu legen, 
betonte bereits 2003 die National Commission on Writing in America’s Schools \& Colleges\cite{nationalcommissionwriting}. \\ Der Prozess des wissenschaftlichen Schreibens stellt variable 
Anforderungen an einen Schüler oder Studenten. Die Fähigkeit, sich selbst zu organisieren und eigenes oder durch Recherche erlangtes Wissen zu strukturieren, ist essenziell. Durch das 
Schreiben wissenschaftlicher Arbeiten trainieren Schüler und Studenten ihre Fähigkeit, Ideen und Konzepte verständlich auszuformulieren. Dies fördert ebenfalls die Fähigkeit, klar zu 
kommunizieren, wodurch bessere Zusammenarbeit im Team gewährleistet wird.\cite{nationalcommissionwriting,teachers,humanWritingToAi} \\ 
Während der Recherche zu wissenschaftlichen Themen wird zudem die Medienkompetenz gestärkt. Zu den häufig verwendeten Medien zählen heute nicht nur Suchmaschinen wie Google Scholar, sondern 
auch diverse Anwendungen, die \gls{ki} nutzen.\\ Nutzen Schüler und Studenten solche KI-gestützten Hilfsmittel, sammeln sie Erfahrungen, die später im Berufsleben hilfreich sein können. Laut 
einer Mitteilung des statistischen Bundesamtes vom 25. November 2024 nutzt bereits jedes fünfte Unternehmen KI, und der Anteil wird voraussichtlich weiter steigen\cite{statistischesBundesamt}. 
Die Technologie wird meist zur Analyse von Schriftsprache sowie zur Erzeugung natürlicher Sprache eingesetzt\cite{statistischesBundesamt}.\\ 
Spätestens seit der Veröffentlichung des verbesserten \glslink{chatgpt}{ChatGPT} von OpenAI im Jahr 2022, das Texte generieren kann, die menschengeschriebenen ähnlich sind, wird die Verwendung von 
\glslink{glos:ki}{KI} im wissenschaftlichen Schreiben viel diskutiert und untersucht.\cite{humanWritingToAi,ZukunftWissenschaftlichesPublizieren}\\
Es zeigen sich Vor- und Nachteile, die durch den Einsatz von \glslink{glos:ki}{KI} an verschiedenen Punkten des Schreibprozesses entstehen. Diese Studienarbeit befasst sich mit diesen 
Vor- und Nachteilen sowie mit möglichen Lösungsvorschlägen zum Umgang mit den Nachteilen der \glslink{glos:ki}{KI}-Nutzung. Insbesondere wird ein neues Schreibassistenten-Programm vorgestellt, das 
die Risiken von \glslink{glos:ki}{KI} minimieren und den Einsatz im Schreibprozess vereinfachen soll. Dazu werden zunächste mehrere Studien beleuchtet und anschließend die erarbeiteten Anforderungen an ein KI-Schreibwerkzeug
umgesetzt und das Ergebnis bewertet.

\end{document}
