\documentclass[../main.tex]{subfiles}
\begin{document}

Wissenschaftliches Schreiben ist ein essenzieller Teil des Schüler- und Studentendaseins. 
Die Relevanz, in Schulen Wert auf die Entwicklung der Schreibfertigkeiten der Schüler zu legen, betonte bereits 2003 
die National Commission on Writing in America’s Schools \& Colleges. Dabei stellt der Prozess des 
wissenschaftlichen Schreibens variable Anforderungen an einen Schüler. Die Fähigkeit, sich selbst zu organisieren und 
eigenes oder durch Recherche erlangtes Wissen zu strukturieren, sind dabei essenziell. Durch das Schreiben 
wissenschaftlicher Arbeiten trainieren Schüler und Studenten ihre Fähigkeit, Ideen und Konzepte verständlich 
auszuformulieren. Dies fördert ebenfalls die Fähigkeit klar zu kommunizieren, wodurch bessere Zusammenarbeit 
im Team gewährleistet wird.  

Zudem wird während der Recherche zu einem wissenschaftlichen Thema die Medienkompetenz gestärkt. Zu den häufig 
verwendeten Medien gehören seit kurzem nicht nur bekannte akademische Suchmaschinen wie Google Scholar, sondern auch 
diverse Tools, welche künstliche Intelligenz (KI) nutzen. Spätestens seit der Veröffentlichung des verbesserten ChatGPT von openAI im Jahr 2022, welches in der 
Lage ist, Texte zu generieren, die menschengeschriebenen sehr ähnlich sind, wird die Verwendung künstlicher Intelligenz 
im wissenschaftlichen Schreiben viel diskutiert und untersucht.\cite{humanWritingToAi}

Dabei zeigen sich sowohl Vor- und Nachteile, welche durch den Einsatz von künstlicher Intelligenz an 
verschiedensten Punkten des Schreibprozesses entstehen. 
Dieses Paper befasst sich sowohl mit diesen Vor- und Nachteilen, als auch mit möglichen
Lösungsvorschlägen zum Umgang mit den Nachteilen der KI-Nutzung. Es wird ein neues Schreibassistenten-Tool vorgestellt, welches die Risiken 
von KI minimieren und den Einsatz von künstlicher Intelligenz im Schreibprozess vereinfachen soll.
\end{document}
