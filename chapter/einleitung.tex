\documentclass[../main.tex]{subfiles}
\begin{document}

Wissenschaftliches Schreiben ist ein essenzieller Teil des Schüler- und Studentendaseins. 
Die Relevanz, in Schulen Wert auf die Entwicklung der Schreibfertigkeiten von Schülern zu legen, betonte bereits 2003 
die National Commission on Writing in America’s Schools \& Colleges\cite{nationalcommissionwriting}.  \\ Der Prozess des 
wissenschaftlichen Schreibens stellt variable Anforderungen an einen Schüler oder Studenten. Die Fähigkeit, sich selbst zu organisieren und 
eigenes oder durch Recherche erlangtes Wissen zu strukturieren, sind essenziell. Durch das Schreiben 
wissenschaftlicher Arbeiten trainieren Schüler und Studenten ihre Fähigkeit, Ideen und Konzepte verständlich 
auszuformulieren. Dies fördert ebenfalls die Fähigkeit klar zu kommunizieren, wodurch bessere Zusammenarbeit 
im Team gewährleistet wird.\cite{nationalcommissionwriting,teachers,humanWritingToAi} \\ 
Zudem wird während der Recherche zu einem wissenschaftlichen Thema die Medienkompetenz gestärkt. Zu den häufig 
verwendeten Medien gehören seit einigen Jahren nicht nur bekannte akademische Suchmaschinen wie Google Scholar, sondern auch 
diverse Anwendungen, welche \gls{ki} nutzen. Spätestens seit der Veröffentlichung des verbesserten ChatGPT von openAI im Jahr 2022, welches in der 
Lage ist, Texte zu generieren, die menschengeschriebenen ähnlich sind, wird die Verwendung von \glslink{glos:ki}{KI} 
im wissenschaftlichen Schreiben viel diskutiert und untersucht.\cite{humanWritingToAi,ZukunftWissenschaftlichesPublizieren}\\
Dabei zeigen sich Vor- und Nachteile, welche durch den Einsatz von \glslink{glos:ki}{KI} an verschiedensten Punkten des Schreibprozesses entstehen. 
Diese Studienarbeit befasst sich sowohl mit diesen Vor- und Nachteilen, als auch mit möglichen
Lösungsvorschlägen zum Umgang mit den Nachteilen der \glslink{glos:ki}{KI}-Nutzung. Es wird ein neues Schreibassistenten-Programm vorgestellt, welches die Risiken 
von \glslink{glos:ki}{KI} minimieren und den Einsatz von \glslink{glos:ki}{KI} im Schreibprozess vereinfachen soll.
\end{document}
